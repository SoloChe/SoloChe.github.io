\documentclass[10pt]{article}
\usepackage[margin = 1in]{geometry}
\usepackage{hyperref}
\title{\bf  Yiming Che \vspace{-0.6em}}

% \author{+1(607)-338-8871 | yche1@binghamton.edu | \href{https://soloche.github.io}{homepage} | \href{https://github.com/SoloChe}{github}}

\author{+1(607)-338-8871 | yche1@binghamton.edu | soloche.github.io | github.com/SoloChe}

\date{}
\usepackage{titling}
\setlength{\droptitle}{-5em} 

\usepackage{titlesec}

\titlespacing*{\section}
{0pt}{5.5ex plus 1ex minus .2ex}{4.3ex plus .2ex}

\usepackage{sectsty}
\sectionfont{\large \sectionrule{0ex}{0pt}{-0.5ex}{0.5pt}}


\usepackage{enumitem}
\usepackage{pdfpages}
\hypersetup{
	colorlinks=true,
	linkcolor=blue,
	filecolor=magenta,      
	urlcolor=cyan,
	pdftitle={Sharelatex Example},
	bookmarks=true,
	pdfpagemode=FullScreen
}
\renewcommand{\baselinestretch}{1.0}

\begin{document}
	\maketitle

\vspace{-5em}

\section*{PROFESSIONAL SKILLS \& KNOWLEDGE}
 \begin{itemize}	
 	\item {\bf Programming Languages:} Python, Matlab, R
 	\item {\bf Skills:} Linux, Slurm, Git/GitHub, MySQL, Pandas, Bash script, PyTorch/TensorFlow, SKlearn
 	\item {\bf Research Interests:} Generative models, Medical imaging, Bayesian statistics, Active learning
 \end{itemize}

\section*{EDUCATION BACKGROUND}
\begin{itemize}
		\item \noindent{\bf Binghamton University, State University of New York, NY, United States
		\\Department of Systems Science and Industrial Engineering}                                                           
		\\Doctor of Philosophy in Systems Science \hfill May 2023 
		\\Advisor: Dr. Changqing Cheng

	\item \noindent{\bf Binghamton University, State University of New York, NY, United States
		\\Department of Systems Science and Industrial Engineering}                                                           
	\\Master of Science in Industrial Engineering \hfill  May 2018\\
	Advisor: Dr. Changqing Cheng
	
    \item \noindent{\bf Capital University of Economics and Business, Beijing, China
         	\\Department of Industrial Engineering}
            \\Bachelor of Science in Industrial Engineering \hfill July 2017 
\end{itemize}


\section*{PROFESSIONAL EXPERIENCE}
\begin{itemize}
	\item \noindent {\bf Postdoctoral Scholar at Arizona State University} (Advised by Dr. Teresa Wu)\\
        \hfill{Aug. 2023 - Present, Tempe, AZ}
        \begin{itemize}[label=$\bullet$]
            \item \noindent {\bf Generative Models on Medical imaging}
            \begin{itemize}[label=$-$]
				\item Developed a fully weakly-supervised brain tumor segmentation framework using guided diffusion models
				\item Developed a high-resolution PET image synthesis strategy using latent diffusion models
				\item Adopted Cycle-GAN model for harmonizing FBP and PiB tracer in amyloid PET
			\end{itemize}

            
            \item \noindent {\bf Interpretable Medical Image Classification}
            \begin{itemize}[label=$-$]
				\item Trying to utilize diffusion models as interpretable medical image classifier with conterfactual 
			\end{itemize}
            \item \noindent {\bf Multi-modality Models for Headache Diagnosis}
            \begin{itemize}[label=$-$]
				\item Utilized BioMedCLIP model for in headache diagnosis and subtype discovery using patient's MRI and clinical notes
			\end{itemize}
        \end{itemize}
        
        
	\item \noindent {\bf Research Assistant at Binghamton University} (Advised by Dr. Changqing Cheng)\\
	\hfill{Feb. 2019 - May 2023, Binghamton, NY}
	\begin{itemize}[label=$\bullet$]
	
	    \item \noindent {\bf Physics-informed Neural Network (PINN) for Covid-19 Outbreak Prediction}
		\begin{itemize}[label=$-$]
	        \item Included Bayesian framework in traditional PINN for robust prediction
		\end{itemize}
		
	    \item \noindent {\bf Surrogate Modeling and Active Learning/Sequential Design}
	    \begin{itemize}[label=$-$]
	        \item Developed a novel surrogate model which combines generalized polynomial chaos and stochastic kriging model for efficient surrogate modeling of stochastic systems
	       
	        \item Developed single-section and batch-selection sampling algorithms with Gaussian process
	    \end{itemize}
	    
		\item \noindent {\bf Uncertainty Quantification for Machining Process}
		\begin{itemize}[label=$-$]
		    \item Developed uncertainty quantification framework using generalized polynomial chaos expansion for machining process 
		\end{itemize}
	\end{itemize}
\end{itemize}


%\vspace{3mm}
%\noindent{\bf AWARD \& HONOR}
\section*{AWARD \& HONOR}
\begin{itemize}
	\item 2023 Distinguished Dissertation Award, Binghamton University
	\hfill{2024}
    \item Excellence in Systems Science Research Award, Binghamton University
	\hfill{2023}
	\item INFORMS Bonder Foundation Award
	\hfill{2021}
	\item Finalist, IISE-DAIS Mobile App Competition at 2021 IISE Annual Conference and Expo
	\hfill{2021}
	\item Binghamton University Graduate Student Excellence Award in Research \textbf{(top 1\%)}
	\hfill{2021}
	\item Travel Grant of Midwest Dynamical Systems Conference 2019, University of Illinois at Chicago 
	\hfill{2019}
	\item Second Place, Best Student Paper Competition at 2019 IISE Annual Conference and Expo \\(Healthcare track)
	\hfill{2019}
	\item Honorable Mention, Binghamton University Research Day Poster Competition, 2018
	\hfill{2018}
	\item National Scholarship, Capital University of Economics and Business
	\hfill{2015}
\end{itemize}


\section*{PUBLICATIONS}
\begin{enumerate}
	\item Shah, J., {\bf Che, Y.}, Sohankar, J., Luo, J., Li, B., Su, Y. and Wu, T. ``Enhancing PET Quantification: MRI-guided Super-Resolution using Latent Diffusion Models'' \textit{Life} (Under review)
	
	\item Shah, J., Krell-Roeschc, J., {\bf Che, Y.}, Forzanie, E., Knopmanf, D.S., Cliff, R.J., Petersenc, R.C., Wu, T. and Geda, Y.E. ``Predicting cognitive decline from neuropsychiatric symptoms and Alzheimer’s disease biomarkers: A machine learning approach to a population-based data'' \textit{Journal of Alzheimer's Disease} (In press)
	
	\item {\bf Che, Y.}, Rafsani, F., Shah, J., Siddiquee, M. M. R. and Wu, T. ``AnoFPDM: Anomaly segmentation with forward Process of diffusion models for brain MRI" \url{https://arxiv.org/abs/2404.15683} (Submitted to ASTAD workshop at WACV 2025).

	\item Wan, J., Kataoka, J., Sivakumar, J., Pena, E., {\bf Che, Y.}, Sayama, H. and Cheng, C. ``Sparse Bayesian learning for sequential inference of network connectivity from Small Data'' \textit{IEEE Transactions on Network Science and Engineering} (In press)
	
	\item {\bf Che, Y.}, Muller, J. and Cheng, C. ``Dispersion-enhanced sequential batch sampling for contour estimation,'' \textit{Quality and Reliability Engineering International} 40 (2024): 131–144. \url{https://doi.org/10.1002/qre.3245} 

	\item {\bf Che, Y.} and Cheng, C. ``Physical-statistical learning towards resilience assessment for power generating systems,'' \textit{Physica A: Statistical Mechanics and its Applications} 615 (2023): 128584. \url{https://doi.org/10.1016/j.physa.2023.128584}

	\item Ma, Q., {\bf Che, Y.}, Cheng, C. and Wang, Z. ``Characterizations and optimization for resilient manufacturing systems with considerations of process uncertainties,'' \textit{Journal of Computing and Information Science in Engineering} 23.1 (2023): 1-30. \url{ https://doi.org/10.1115/1.4055425}

	\item Wan, J., {\bf Che, Y.}, Wang, Z. and Cheng, C. ``Uncertainty quantification and optimal robust design for machining operations,'' \textit{Journal of Computing and Information Science in Engineering} 23.1 (2023): 0110005. \url{https://doi.org/10.1115/1.4055039}

	\item {\bf Che, Y.} and Cheng, C. ``Active learning and relevance vector machine in efficient estimate for basin stability of dynamic networks,'' \textit{Chaos: An Interdisciplinary Journal of Nonlinear Science} 31.5 (2021): 053129. \url{https://doi.org/10.1063/5.0044899}.

	\item {\bf Che, Y.}, Guo, Z. and Cheng, C. ``Generalized polynomial chaos-informed efficient stochastic Kriging,'' \textit{Journal of Computational Physics} 445 (2021): 110598. \url{https://doi.org/10.1016/j.jcp.2021.110598}.

	\item Wu, X., Zheng, Y., {\bf Che, Y.} and Cheng, C. ``Pattern recognition and automatic identification of early-stage atrial fibrillation,''\textit{Expert Systems with Applications} 158 (2020): 113560. \url{https://doi.org/10.1016/j.eswa.2020.113560}.

	\item {\bf Che, Y.}, Cheng, C., Liu, Z. and Zhang, Z. ``Fast basin stability estimation for dynamic systems under large perturbations with sequential support vector machine,''\textit{Physica D: Nonlinear Phenomena} 405 (2020): 132381. \url{https://doi.org/10.1016/j.physd.2020.132381}.

	\item {\bf Che, Y.}, Liu, J.  and Cheng, C. ``Multi-fidelity modeling in sequential design for identification of stability region in dynamic time-delay systems,''  \textit{Chaos: An Interdisciplinary Journal of Nonlinear Science} 29.9 (2019): 093-105. \url{https://doi.org/10.1063/1.5097934}.

	\item {\bf Che, Y.} and Cheng, C. ``Uncertainty quantification in stability analysis of chaotic systems with discrete delays,'' \textit{Chaos, Solitons} \& \textit{Fractals} 116 (2018): 208-214. \url{https://doi.org/10.1016/j.chaos.2018.08.024}.
\end{enumerate}

\section*{\bf PROFESSIONAL SERVICES}
\begin{itemize}	
		\item Conference Reviewer
		\begin{itemize}[label=$\bullet$]
			\item Medical Image Computing and Computer Assisted Intervention (MICCAI)
			\item International Conference on Automation Science and Engineering (CASE)
		\end{itemize}
        \item Journal Reviewer 
        \begin{itemize}[label=$\bullet$]
            \item Physica D: Nonlinear Phenomena
            \item Scientific Reports
        \end{itemize}
        
	\item Student member, Student leadership board at IISE
	\hfill{Mar. 2021-2022}
	\item Vice president, ASQ Binghamton chapter
	\hfill{Aug. 2019-2022}
\end{itemize}

% \section*{\bf Bio}
% Yiming Che, Ph.D., is a postdoctoral research Scholar in the Wu Lab at Arizona State University, under the mentorship of Dr. Teresa Wu. He earned his Ph.D. in Systems Science from Binghamton University, where he was advised by Dr. Changqing Cheng. His research focuses on the application of generative models in medical imaging, Bayesian statistics, Gaussian processes, and active learning. Dr. Che has authored over 10 peer-reviewed publications and has been recognized with several prestigious awards, including the 2023 Distinguished Dissertation Award, the 2023 Excellence in Systems Science Research Award, and the 2021 Graduate Student Excellence Award in Research from Binghamton University.



\end{document}