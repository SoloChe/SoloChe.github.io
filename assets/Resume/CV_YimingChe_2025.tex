\documentclass[10pt]{article}
\usepackage[margin = 0.5in]{geometry}
\usepackage{hyperref}
\title{\bf  Yiming Che \vspace{-0.4em}}

% \author{+1(607)-338-8871 | yche1@binghamton.edu | \href{https://soloche.github.io}{homepage} | \href{https://github.com/SoloChe}{github}}

\author{+1(607)338-8871 $\vert$ yche1@binghamton.edu $\vert$ \href{https://soloche.github.io}{Homepage} $\vert$ \href{https://scholar.google.com/citations?user=HFI1L0QAAAAJ}{Google Scholar} $\vert$ \href{https://github.com/SoloChe}{GitHub}}

\date{}
\usepackage{titling}
\setlength{\droptitle}{-5em} 

\usepackage{titlesec}

\titlespacing*{\section}
{0pt}{5.5ex plus 1ex minus .2ex}{4.3ex plus .2ex}

\usepackage{sectsty}
\sectionfont{\large \sectionrule{0ex}{0pt}{-0.5ex}{0.5pt}}


\usepackage{enumitem}
\usepackage{pdfpages}
\hypersetup{
	colorlinks=true,
	linkcolor=blue,
	filecolor=magenta,      
	urlcolor=cyan,
	pdftitle={Sharelatex Example},
	bookmarks=true,
	pdfpagemode=FullScreen
}
\renewcommand{\baselinestretch}{1.0}

\begin{document}
	\maketitle

\vspace{-5em}

\section*{PROFESSIONAL SKILLS \& KNOWLEDGE}
\vspace{-0.5em}
 \begin{itemize}
	\setlength\itemsep{0.5pt}
	% \item \textbf{Summary:} Data Scientist and Machine Learning Researcher with a PhD in Systems Science, 8+ years of experience in statistical modeling, generative AI, and healthcare analytics. Adept at solving real-world problems through large-scale data analysis, predictive modeling, and optimization. Experienced in cross-functional collaboration with clinicians, engineers, and researchers.
	% \vspace{-0.4em}
 	\item {\bf Programming Languages:} Python, Matlab, R
 	\vspace{-0.4em}
 	\item {\bf ML Frameworks:} PyTorch, TensorFlow, scikit-learn, HuggingFace, LangChain
 	\vspace{-0.4em}
 	\item {\bf Data Pipeline:} SQL, Pandas, PySpark, Snowflake, Databricks
 	\vspace{-0.4em}
 	\item {\bf Cloud/DevOps:} AWS, SageMaker, GCP, Airflow, Docker, CI/CD, MLflow, Flask/FastAPI
 	\vspace{-0.4em}
	\item {\bf Tools:} Linux, Git/GitHub, Bash
 	\vspace{-0.4em}
 \end{itemize}

\vspace{-1.5em}
\section*{EDUCATION BACKGROUND}
\vspace{-0.5em}

\begin{itemize}
	\setlength\itemsep{0.5pt}
	\item \noindent{\bf Binghamton University, State University of New York, NY, United States}                                                           
	\\Doctor of Philosophy in Systems Science (focus on Machine Learning) \hfill May 2023 
	% \\Advisor: Dr. Changqing Cheng
	\vspace{-0.4em}
	\item \noindent{\bf Binghamton University, State University of New York, NY, United States}                                                           
	\\Master of Science in Industrial Engineering \hfill  May 2018
	% Advisor: Dr. Changqing Cheng
	\vspace{-0.4em}
    \item \noindent{\bf Capital University of Economics and Business, Beijing, China}
    \\Bachelor of Science in Industrial Engineering \hfill July 2017 
	\vspace{-0.3em}
\end{itemize}

\vspace{-1.5em}
\section*{PROFESSIONAL EXPERIENCE}
\vspace{-0.5em}

\begin{itemize}
	\setlength\itemsep{0.5pt}
	\item \noindent {\bf Senior ML/Data Scientist at Walmart Global Tech}\\
		\hfill{Sep. 2025 - Present, Bentonville, AR}
		\vspace{-0.5em}
		\begin{itemize}[label=$\bullet$]
			\setlength\itemsep{0.5pt}
			\item \noindent {\bf Led Development of Generative AI Solutions for Recommendation Systems (In Progress)}
			\vspace{-0.5em}
			\begin{itemize}[label=$\bullet$]
				\setlength\itemsep{0.5pt}
				\item Developed campaign-2-PT RAG system using \textbf{PyTorch} and \textbf{HuggingFace} to retrieve campaign-related products for campaign labeling for supervised ranker model.
				\item Explored and analyzed large-scale campaign and purchase datasets using \textbf{PySpark}.
			\end{itemize}

			\item \noindent {\bf Co-led Development of Campaign Push/Notification (In Progress)}
			\vspace{-0.5em}
			\begin{itemize}[label=$\bullet$]
				\item Developed \textbf{PySpark} pipeline to monitor campaign performance.
			\end{itemize}
		\end{itemize}

	\item \noindent {\bf Postdoctoral Scholar at Arizona State University \\ Research Scientist at ASU-Mayo Center for Innovative Imaging (Concurrent)}\\
        \hfill{July 2023 - Sep. 2025, Tempe, AZ}
		\vspace{-0.5em}
        \begin{itemize}[label=$\bullet$]

			\setlength\itemsep{0.5pt}
			\item \noindent \textbf{Led Development of End-to-End Cycle-GAN for Tracer Data Translation (In Progress)}
			\vspace{-0.5em}
			\begin{itemize}[label=$\bullet$]
				\setlength\itemsep{0.5pt}
				\item Optimized Cycle-GAN model for translation between FBP and PiB tracer (tabular data) in amyloid PET images. \textbf{Eliminated the requirement of paired data without loss of accuracy (achieved correlation 0.97)}. Reduced clinical trial costs and expand access to amyloid imaging in non-specialist settings. [\href{https://github.com/SoloChe/e2e_PET_tracer_translation}{Project Link}] (Paper in preparation)
				\item Utilized \textbf{Airbyte} for data extraction from \textbf{AWS S3} to \textbf{Snowflake} for \textbf{data ELT}, and performed data \textbf{EDA} using \textbf{Pandas}.
				\item Implemented modified Cycle-GAN model using \textbf{PyTorch} and deployed the \textbf{dockerized} model using \textbf{AWS SageMaker} and \textbf{Flask} for inference. Utilized GitHub Action for \textbf{CI/CD}.
				\item Collaborated with clinicians from Banner Alzheimer’s Institute to validate model outputs.
			\end{itemize}

			\item \noindent \textbf{Led Development of End-to-End Multi-agent Medical Q\&A System}
			\vspace{-0.5em}
			\begin{itemize}[label=$\bullet$]
			\setlength\itemsep{0.5pt}
				\item Implemented medical Q\&A system with multi-agent retrieval-augmented generation (RAG) using \textbf{HuggingFace} and \textbf{PyTorch} to build fully customized clinical support tools. [\href{https://github.com/SoloChe/Medical_QA_RAG}{Project Link}]
				\item Vectorized medical datasets for corpus using \textbf{FAISS}.
				\item Fine-tuned LLMs with 7B parameter with \textbf{LoRa} utilizing distributed training. 
				\item Deployed the inference pipeline using \textbf{AWS SageMaker} and \textbf{Flask} for real-time inference.
				% \item Optimizing the RAG with \textbf{ReAct} using \textbf{LangChain} and \textbf{OpenAI API} to enhance the performance.
			\end{itemize}

			\setlength\itemsep{0.5pt}
            \item \noindent {\bf Led Development of Diffusion Models on Medical Imaging}
            \vspace{-0.5em}
            \begin{itemize}[label=$\bullet$]
				\item Developed a fully weakly-supervised anomaly detection/segmentation framework (\href{https://openaccess.thecvf.com/content/WACV2025W/ASTAD/papers/Che_AnoFPDM_Anomaly_Detection_with_Forward_Process_of_Diffusion_Models_for_WACVW_2025_paper.pdf}{AnoFPDM})  using guided diffusion models. \textbf{Achieved state-of-the-art performance on lesion segmentation with DICE score 77.4 on BraTS21 dataset}, eliminating pixel-level labels for hyperparameter tuning, which significantly reduces the annotation cost.
				% \item Developed a high-resolution PET image synthesis pipeline  using latent diffusion models and low-dose PET image (\href{https://www.mdpi.com/2075-1729/14/12/1580}{paper}). Achieved correlation 0.94 between synthesized PET image and digital phantoms. Reduced the need for high-dose PET scans, minimized patient radiation exposure and enabled low-dose PET imaging to achieve high-resolution results, making scans more accessible and affordable.
				\item Implemented various diffusion models in \textbf{PyTorch} using \textbf{distributed training/inference} on Linux (Slurm job scheduler). 
			\end{itemize}

			\item \noindent {\bf Led Development of Fusion of CT and MRI for Traumatic Brain Injury Recovery Prediction} 
			\vspace{-0.5em}
			\begin{itemize}[label=$\bullet$]
				\setlength\itemsep{0.5pt}
				\item Utilized Cycle-GAN to generate synthetic MRI from real CT to address long waiting time of MRI. Developed a multi-modal classification pipeline combining CT and synthetic MRI using ResNet. \textbf{Achieved $\sim$16\% AUC improvement} compared to only using single CT modality. 
				\item Collaborated with clinicians at Mayo Clinic to validate model outputs, ensuring the solution addressed real patient-care needs. (Paper under review)
			\end{itemize}

			\item \noindent {\bf Co-led Development of Machine Learning for Cognitive Decline Prediction}
			\vspace{-0.5em}
			\begin{itemize}[label=$\bullet$]
				\item Conducted \textbf{model selection} from classification models, e.g., \textbf{XGBoost, random forest and SVM}, with nested cross-validation for robust cognitive decline prediction.
				\item Applied \textbf{SHAP analysis} and Wald test for feature importance in cognitive decline prediction. \textbf{Identified top 5 features}, providing insights to the clinical research. (\href{https://journals.sagepub.com/doi/abs/10.1177/13872877241306654}{paper})
				\item Collaborated cross-functionally with clinicians and data engineers from Mayo Clinic to preprocess patient cognitive assessments.
			\end{itemize}

			\item \noindent {\bf Co-led Development of Multi-modality (Text and Image) Models for Headache Diagnosis}
            \vspace{-0.5em}
            \begin{itemize}[label=$\bullet$]
				\setlength\itemsep{0.5pt}
				\item Fine-tuned multi-modal classification pipelines combining MRI and clinical notes based on BioMedCLIP using \textbf{PyTorch}. Fine-tuned solely on PubMedBERT and ViT for co-learning. \textbf{Achieved state-of-the-art performance in headache diagnosis with 0.96 AUC}. Reduced misdiagnosis rates, potentially saving hospitals and insurance companies on unnecessary treatments. (Paper under review)
				\item Collaborated cross-functionally with clinicians from Mayo Clinic for biomarker extraction and clinical interpretation.
			\end{itemize}
        
	\end{itemize}
		
	
	\item \noindent {\bf Research Assistant (PhD) at Binghamton University} \\
		\hfill{Aug. 2017 - May 2023, Binghamton, NY}
		\vspace{-0.5em}
		\begin{itemize}[label=$\bullet$]
			\setlength\itemsep{0.5pt}
			\item \noindent {\bf Researched Bayesian Statistics and Uncertainty Quantification}
			\vspace{-0.5em}
			\begin{itemize}[label=$\bullet$]
				\setlength\itemsep{0.5pt}
				\item Integrated a Bayesian framework into traditional PINN for enhanced robustness and uncertainty quantification. Provided confidence intervals for predictions and improved reliability over non-Bayesian PINNs for more trustworthy decision-making process. 
				\item Developed a novel Bayesian surrogate model which combines generalized polynomial chaos and Gaussian process for efficient surrogate modeling of stochastic systems. Achieved $\sim$90\% improvement in computational budget without loss of accuracy compared to traditional Monte Carlo simulation.
				\item Developed single-section and batch-selection sampling algorithms with Gaussian process. Achieved $\sim$70\% improvement in computational efficiency compared to traditional one-shot design.
				\item Developed uncertainty quantification framework using generalized polynomial chaos expansion for machining process. Achieved $\sim$80\% improvement in computational efficiency compared to Monte Carlo simulation.
			\end{itemize}
	\end{itemize}
\end{itemize}

\vspace{-1.5em}
\section*{SELECTED AWARD \& HONOR}
\vspace{-0.5em}
\begin{itemize}
	\setlength\itemsep{0.5pt}
	\item Distinguished Dissertation Award, Binghamton University \textbf{(top 1\%)}
	\hfill{2024}
	\vspace{-0.4em}
    \item Excellence in Systems Science Research Award, Binghamton University \textbf{(top 1\%)}
	\hfill{2023}
	% \item INFORMS Bonder Foundation Award
	% \hfill{2021}
	% \item Finalist, IISE-DAIS Mobile App Competition at 2021 IISE Annual Conference and Expo
	% \hfill{2021}
	\vspace{-0.4em}
	\item Binghamton University Graduate Student Excellence Award in Research \textbf{(top 1\%)}
	\hfill{2021}
	\vspace{-0.3em}
\end{itemize}

\vspace{-1.5em}
\section*{SELECTED PUBLICATIONS} 
\vspace{-0.5em}
\hspace{1.2em}\textbf{Summary:} 18 publications (10 first-authored), 3 working papers, 138 citations, h-index: 8, i10-index: 8 (as of 12-15-2025)\\
\vspace{-1.3em}
\begin{enumerate}
	\setlength\itemsep{0.5pt}
	\item {\bf Che, Y.}, Rafsani, F., Shah, J., Siddiquee, M. M. R. and Wu, T. ``AnoFPDM: Anomaly segmentation with forward process of diffusion models for brain MRI" \textit{Proceedings of the Winter Conference on Applications of Computer Vision. 2025.} \url{https://arxiv.org/abs/2404.15683}
	\vspace{-0.4em}
	\item Wan, J., Kataoka, J., Sivakumar, J., Pena, E., {\bf Che, Y.}, Sayama, H. and Cheng, C. ``Sparse Bayesian learning for sequential inference of network connectivity from Small Data'' \textit{IEEE Transactions on Network Science and Engineering} 11.6 (2024): 5892-5902. \url{https://doi.org/10.1109/TNSE.2024.3471852}
	\vspace{-0.4em}
	\item {\bf Che, Y.}, Guo, Z. and Cheng, C. ``Generalized polynomial chaos-informed efficient stochastic Kriging,'' \textit{Journal of Computational Physics} 445 (2021): 110598. \url{https://doi.org/10.1016/j.jcp.2021.110598}
	\vspace{-0.4em}
	\item {\bf Che, Y.} and Cheng, C. ``Uncertainty quantification in stability analysis of chaotic systems with discrete delays,'' \textit{Chaos, Solitons} \& \textit{Fractals} 116 (2018): 208-214. \url{https://doi.org/10.1016/j.chaos.2018.08.024}
\end{enumerate}

% \begin{enumerate}
% 	\setlength\itemsep{0.5pt}
% 	\item {\bf Che, Y.}, Rafsani, F., Shah, J., Siddiquee, M. M. R. and Wu, T. ``AnoFPDM: Anomaly segmentation with forward process of diffusion models for brain MRI" \url{https://arxiv.org/abs/2404.15683} (Accepted by ASTAD workshop at WACV 2025 for oral presentation).

% 	\item Shah, J., {\bf Che, Y.}, Sohankar, J., Luo, J., Li, B., Su, Y. and Wu, T. ``Enhancing PET quantification: MRI-guided super-resolution using latent diffusion models'' \textit{Life} 14.12 (2024): 1580. \url{https://doi.org/10.3390/life14121580}
	
% 	\item Shah, J., Krell-Roeschc, J., Forzanie, E., Knopmanf, D.S., Cliff, R.J., Petersenc, R.C., {\bf Che, Y.}, Wu, T. and Geda, Y.E. ``Predicting cognitive decline from neuropsychiatric symptoms and Alzheimer’s disease biomarkers: A machine learning approach to a population-based data'' \textit{Journal of Alzheimer's Disease} 13872877241306654 (2025). \url{https://journals.sagepub.com/doi/full/10.1177/13872877241306654}
	
% 	\item Wan, J., Kataoka, J., Sivakumar, J., Pena, E., {\bf Che, Y.}, Sayama, H. and Cheng, C. ``Sparse Bayesian learning for sequential inference of network connectivity from Small Data'' \textit{IEEE Transactions on Network Science and Engineering} 11.6 (2024): 5892-5902. \url{https://doi.org/10.1109/TNSE.2024.3471852}
	
% 	\item {\bf Che, Y.}, Muller, J. and Cheng, C. ``Dispersion-enhanced sequential batch sampling for contour estimation,'' \textit{Quality and Reliability Engineering International} 40 (2024): 131–144. \url{https://doi.org/10.1002/qre.3245} 

% 	\item {\bf Che, Y.} and Cheng, C. ``Physical-statistical learning towards resilience assessment for power generating systems,'' \textit{Physica A: Statistical Mechanics and its Applications} 615 (2023): 128584. \url{https://doi.org/10.1016/j.physa.2023.128584}

% 	\item Ma, Q., {\bf Che, Y.}, Cheng, C. and Wang, Z. ``Characterizations and optimization for resilient manufacturing systems with considerations of process uncertainties,'' \textit{Journal of Computing and Information Science in Engineering} 23.1 (2023): 1-30. \url{ https://doi.org/10.1115/1.4055425}

% 	\item Wan, J., {\bf Che, Y.}, Wang, Z. and Cheng, C. ``Uncertainty quantification and optimal robust design for machining operations,'' \textit{Journal of Computing and Information Science in Engineering} 23.1 (2023): 0110005. \url{https://doi.org/10.1115/1.4055039}

% 	\item {\bf Che, Y.} and Cheng, C. ``Active learning and relevance vector machine in efficient estimate for basin stability of dynamic networks,'' \textit{Chaos: An Interdisciplinary Journal of Nonlinear Science} 31.5 (2021): 053129. \url{https://doi.org/10.1063/5.0044899}.

% 	\item {\bf Che, Y.}, Guo, Z. and Cheng, C. ``Generalized polynomial chaos-informed efficient stochastic Kriging,'' \textit{Journal of Computational Physics} 445 (2021): 110598. \url{https://doi.org/10.1016/j.jcp.2021.110598}.

% 	\item Wu, X., Zheng, Y., {\bf Che, Y.} and Cheng, C. ``Pattern recognition and automatic identification of early-stage atrial fibrillation,''\textit{Expert Systems with Applications} 158 (2020): 113560. \url{https://doi.org/10.1016/j.eswa.2020.113560}.

% 	\item {\bf Che, Y.}, Cheng, C., Liu, Z. and Zhang, Z. ``Fast basin stability estimation for dynamic systems under large perturbations with sequential support vector machine,''\textit{Physica D: Nonlinear Phenomena} 405 (2020): 132381. \url{https://doi.org/10.1016/j.physd.2020.132381}.

% 	\item {\bf Che, Y.}, Liu, J.  and Cheng, C. ``Multi-fidelity modeling in sequential design for identification of stability region in dynamic time-delay systems,''  \textit{Chaos: An Interdisciplinary Journal of Nonlinear Science} 29.9 (2019): 093-105. \url{https://doi.org/10.1063/1.5097934}.

% 	\item {\bf Che, Y.} and Cheng, C. ``Uncertainty quantification in stability analysis of chaotic systems with discrete delays,'' \textit{Chaos, Solitons} \& \textit{Fractals} 116 (2018): 208-214. \url{https://doi.org/10.1016/j.chaos.2018.08.024}.
% \end{enumerate}

% \section*{\bf PROFESSIONAL SERVICES}
% \begin{itemize}	
% 		\item Conference Reviewer
% 		\begin{itemize}[label=$\bullet$]
% 			\item Medical Image Computing and Computer Assisted Intervention (MICCAI)
% 			\item International Conference on Automation Science and Engineering (CASE)
% 		\end{itemize}
%         \item Journal Reviewer 
%         \begin{itemize}[label=$\bullet$]
%             \item Physica D: Nonlinear Phenomena
%             \item Scientific Reports
%         \end{itemize}
        
% 	\item Student member, Student leadership board at IISE
% 	\hfill{Mar. 2021-2022}
% 	\item Vice president, ASQ Binghamton chapter
% 	\hfill{Aug. 2019-2022}
% \end{itemize}
\end{document}